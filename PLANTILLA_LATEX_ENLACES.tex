% ====================================================================
% PLANTILLA LATEX: ENLACES A CÓDIGO EN GITHUB
% Para el libro "Introducción a la Econometría"
% ====================================================================

% ---- CONFIGURACIÓN EN EL PREÁMBULO ----

% Agregar en tu preámbulo:
\usepackage{hyperref}
\usepackage{xcolor}
\usepackage{tcolorbox}
\usepackage{listings}

% Configurar colores para enlaces
\hypersetup{
    colorlinks=true,
    linkcolor=blue,
    filecolor=magenta,      
    urlcolor=blue,
    citecolor=green,
    pdftitle={Introducción a la Econometría},
    pdfauthor={Tu Nombre},
    pdfsubject={Econometría},
    pdfkeywords={econometría, Python, R, regresión},
}

% ====================================================================
% EJEMPLO 1: SECCIÓN DE CÓDIGO AL FINAL DE CADA CAPÍTULO
% ====================================================================

\section{Implementación Computacional}

Todo el código de este capítulo está disponible públicamente en GitHub:

\begin{tcolorbox}[colback=blue!5!white, colframe=blue!75!black, title=📦 Recursos Computacionales]

\textbf{Repositorio completo:}
\begin{center}
\url{https://github.com/tu-usuario/econometria-libro-codigo}
\end{center}

\textbf{Código de este capítulo:}
\begin{itemize}
\item \textbf{Python}: \href{https://github.com/tu-usuario/econometria-libro-codigo/blob/main/capitulo03_algebra_matricial/scripts/verificacion_cap3.py}{\texttt{verificacion\_cap3.py}}

\item \textbf{R}: \href{https://github.com/tu-usuario/econometria-libro-codigo/blob/main/capitulo03_algebra_matricial/scripts/caso_practico.R}{\texttt{caso\_practico.R}}

\item \textbf{Jupyter Notebook}: \href{https://github.com/tu-usuario/econometria-libro-codigo/blob/main/capitulo03_algebra_matricial/notebooks/cap3_interactivo.ipynb}{\texttt{cap3\_interactivo.ipynb}}

\item \textbf{Datos}: \href{https://github.com/tu-usuario/econometria-libro-codigo/blob/main/capitulo03_algebra_matricial/datos/datos_gasto_hogares.csv}{\texttt{datos\_gasto\_hogares.csv}}
\end{itemize}

\textbf{Ejecutar sin instalar nada:}
\begin{itemize}
\item Python en Colab: \href{https://colab.research.google.com/github/tu-usuario/econometria-libro-codigo/blob/main/capitulo03_algebra_matricial/notebooks/cap3_interactivo.ipynb}{Abrir en Google Colab}

\item R en Posit Cloud: \href{https://posit.cloud/content/TU-ID-AQUI}{Abrir en RStudio Cloud}
\end{itemize}

\end{tcolorbox}

% ====================================================================
% EJEMPLO 2: FORMATO COMPACTO (AHORRA ESPACIO)
% ====================================================================

\subsection{Código y Datos}

\textbf{GitHub}: \url{https://github.com/tu-usuario/econometria-libro-codigo/tree/main/capitulo03_algebra_matricial}

Scripts: \href{URL}{\texttt{Python}} | \href{URL}{\texttt{R}} | \href{URL}{\texttt{Jupyter}} | Datos: \href{URL}{\texttt{CSV}}

Ejecutar online: \href{URL}{Colab} | \href{URL}{Posit Cloud}

% ====================================================================
% EJEMPLO 3: TABLA CON TODOS LOS CAPÍTULOS (EN LA INTRODUCCIÓN)
% ====================================================================

\section{Recursos Computacionales del Libro}

\begin{table}[h]
\centering
\caption{Enlaces a código por capítulo}
\begin{tabular}{clcc}
\toprule
\textbf{Cap.} & \textbf{Tema} & \textbf{Python} & \textbf{R} \\
\midrule
1 & Introducción & \href{URL}{📄} & \href{URL}{📄} \\
2 & Estadística Básica & \href{URL}{📄} & \href{URL}{📄} \\
3 & Álgebra Matricial & \href{URL}{📄} & \href{URL}{📄} \\
4 & Regresión Simple & \href{URL}{📄} & \href{URL}{📄} \\
5 & Regresión Múltiple & \href{URL}{📄} & \href{URL}{📄} \\
\vdots & \vdots & \vdots & \vdots \\
18 & Máxima Verosimilitud & \href{URL}{📄} & \href{URL}{📄} \\
\bottomrule
\end{tabular}
\end{table}

Repositorio completo: \url{https://github.com/tu-usuario/econometria-libro-codigo}

% ====================================================================
% EJEMPLO 4: SECCIÓN INLINE DENTRO DEL TEXTO
% ====================================================================

El código completo para este ejemplo está disponible en 
\href{https://github.com/tu-usuario/econometria-libro-codigo/blob/main/capitulo03_algebra_matricial/scripts/verificacion_cap3.py}{\texttt{verificacion\_cap3.py}}, 
donde se puede verificar numéricamente que $R^2 = 0.9994$.

% ====================================================================
% EJEMPLO 5: NOTA AL PIE (FOOTNOTE)
% ====================================================================

El modelo de regresión múltiple\footnote{Código disponible en: 
\url{https://github.com/tu-usuario/econometria-libro-codigo/tree/main/capitulo03_algebra_matricial}} 
produce estimaciones robustas cuando...

% ====================================================================
% EJEMPLO 6: MARGEN LATERAL (REQUIERE PAQUETE MARGINNOTE)
% ====================================================================

\usepackage{marginnote}

% En el texto:
El estimador MCO es...

\marginnote{
\footnotesize
\textbf{Código:}\\
\href{URL}{Python}\\
\href{URL}{R}\\
\href{URL}{Colab}
}

% ====================================================================
% EJEMPLO 7: CÓDIGO QR (PARA VERSIÓN IMPRESA)
% ====================================================================

% Generar código QR online en: https://www.qr-code-generator.com/
% Luego incluir la imagen:

\begin{center}
\includegraphics[width=3cm]{qr_code_cap3.png}\\
\small Escanea para acceder al código
\end{center}

% ====================================================================
% EJEMPLO 8: LISTA DE EJERCICIOS CON CÓDIGO
% ====================================================================

\subsection{Ejercicios Computacionales}

\begin{enumerate}
\item Replique el ejemplo 3.1 usando el código en 
\href{URL}{\texttt{ejemplo3\_1.py}}. Modifique los datos y observe...

\item Complete el ejercicio 3.5 usando como base el código en 
\href{URL}{\texttt{ejercicio3\_5\_template.py}}. Su tarea es...

\item Los datos para este ejercicio están en 
\href{URL}{\texttt{datos\_ejercicio3.csv}}. Estime el modelo...
\end{enumerate}

% ====================================================================
% EJEMPLO 9: APÉNDICE CON INSTRUCCIONES DE INSTALACIÓN
% ====================================================================

\appendix
\chapter{Recursos Computacionales}

\section{Acceso al Código}

\subsection{Opción 1: Repositorio GitHub (Recomendado)}

Todo el código está disponible en:

\begin{center}
\Large\url{https://github.com/tu-usuario/econometria-libro-codigo}
\end{center}

\textbf{Para descargar}:
\begin{lstlisting}[language=bash]
# Clonar el repositorio completo
git clone https://github.com/tu-usuario/econometria-libro-codigo.git

# O descargar como ZIP
# Click en "Code" > "Download ZIP" en GitHub
\end{lstlisting}

\subsection{Opción 2: Google Colab (Sin Instalación)}

Los notebooks Jupyter se pueden ejecutar directamente en Google Colab:

\begin{itemize}
\item \href{URL}{Capítulo 1: Introducción}
\item \href{URL}{Capítulo 2: Estadística Básica}
\item \href{URL}{Capítulo 3: Álgebra Matricial}
\item (etc.)
\end{itemize}

No requiere instalación local. Solo necesitas una cuenta de Google.

\subsection{Opción 3: Posit Cloud (Para R)}

Los scripts de R están disponibles en Posit Cloud:

\begin{itemize}
\item \href{URL}{Proyecto completo en Posit Cloud}
\end{itemize}

RStudio completo en tu navegador.

\section{Instalación Local}

\subsection{Python}

\textbf{Requisitos}: Python 3.8 o superior

\begin{lstlisting}[language=bash]
# Clonar repositorio
git clone https://github.com/tu-usuario/econometria-libro-codigo.git
cd econometria-libro-codigo

# Crear entorno virtual (opcional pero recomendado)
python -m venv env_econometria
source env_econometria/bin/activate  # En Windows: env_econometria\Scripts\activate

# Instalar dependencias
pip install -r requirements.txt

# Ejecutar un ejemplo
cd capitulo03_algebra_matricial/scripts
python verificacion_cap3.py
\end{lstlisting}

\subsection{R}

\textbf{Requisitos}: R 4.0 o superior

\begin{lstlisting}[language=R]
# Instalar paquetes necesarios
install.packages(c("tidyverse", "lmtest", "sandwich", 
                   "car", "tseries", "vars"))

# Ejecutar un script
setwd("capitulo03_algebra_matricial/scripts")
source("caso_practico.R")
\end{lstlisting}

% ====================================================================
% EJEMPLO 10: PÁGINA DE CRÉDITOS CON REPOSITORIO
% ====================================================================

\chapter*{Sobre este Libro}

\section*{Recursos Digitales}

Este libro viene acompañado de:

\begin{itemize}
\item \textbf{Código completo} en Python y R
\item \textbf{Conjuntos de datos} reales y simulados
\item \textbf{Notebooks interactivos} ejecutables en la nube
\item \textbf{Ejercicios resueltos} paso a paso
\end{itemize}

Todo disponible gratuitamente en:

\begin{center}
\Large\textbf{\url{https://github.com/tu-usuario/econometria-libro-codigo}}
\end{center}

\vspace{1cm}

\begin{tcolorbox}[colback=gray!10, colframe=gray!50, title=Cómo usar los recursos]
\begin{enumerate}
\item \textbf{Ver online}: Navega por el repositorio en GitHub
\item \textbf{Descargar}: Click en "Code" > "Download ZIP"
\item \textbf{Clonar}: \texttt{git clone [URL]}
\item \textbf{Ejecutar en la nube}: Click en los badges de Colab/Posit Cloud
\end{enumerate}
\end{tcolorbox}

\section*{Reportar Erratas}

Si encuentras errores en el código o en el libro:

\begin{itemize}
\item \textbf{Issues de GitHub}: \url{https://github.com/tu-usuario/econometria-libro-codigo/issues}
\item \textbf{Email}: \href{mailto:tu.email@ejemplo.com}{tu.email@ejemplo.com}
\end{itemize}

% ====================================================================
% EJEMPLO 11: BADGES PROFESIONALES (REQUIERE INCLUIR IMÁGENES)
% ====================================================================

% Descargar badges de:
% - https://shields.io/
% - https://colab.research.google.com/assets/colab-badge.svg

\begin{center}
\href{https://github.com/tu-usuario/econometria-libro-codigo}{
    \includegraphics[height=0.8cm]{badge_github.png}
}
\hspace{0.5cm}
\href{https://colab.research.google.com/...}{
    \includegraphics[height=0.8cm]{badge_colab.png}
}
\hspace{0.5cm}
\href{https://posit.cloud/...}{
    \includegraphics[height=0.8cm]{badge_rstudio.png}
}
\end{center}

% ====================================================================
% NOTAS FINALES
% ====================================================================

% RECORDATORIOS:
% 1. Reemplazar "tu-usuario" con tu usuario real de GitHub
% 2. Reemplazar "URL" con las URLs completas de tus archivos
% 3. Compilar con pdflatex o xelatex
% 4. Asegurarte de que hyperref está configurado
% 5. Los enlaces funcionarán en la versión PDF digital

% VENTAJAS DE ESTE ENFOQUE:
% ✓ Los lectores hacen clic y acceden directamente al código
% ✓ El código está siempre actualizado (en GitHub)
% ✓ Fácil de mantener (un solo lugar)
% ✓ Profesional y moderno
% ✓ Permite colaboración y feedback

% ====================================================================
